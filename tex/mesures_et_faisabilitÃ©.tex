\documentclass[a4paper]{article}

\usepackage[french]{babel}
\usepackage[T1]{fontenc}
\usepackage[utf8]{inputenc}
\usepackage{lmodern}
\usepackage{microtype}
\usepackage{hyperref}

\author{Maxime WIRTH}
\title{Nouveaux ordres de grandeur}

\begin{document}
\maketitle

Le diamètre du câble est de $D_{câble} = 4.950 * 10^{-3} m$.
Pour $L = 3m$, on doit avoir 202 étages de fil et pour obtenir $3000 tr.m^{-1}$, on doit avoir 15 épaisseurs de fil.
On peut approximer notre solénoïde par un cylindre creux de rayon intérieur $a = 1m$ et de rayon extérieur $R_{ext} = a + e$ avec $e = 15 * D_{câble} = 7.4*10^{-2} m$.
On obtient un volume de $V = 1.45 m^3$.
Avec $\ro = 5.7 * 10^3 kg.m^{-3}$, on obtient finalement $M_{tot} = 8.244 t$.

\end{document}