\documentclass[a4paper]{article}

\usepackage[french]{babel}
\usepackage[T1]{fontenc}
\usepackage[utf8]{inputenc}
\usepackage{lmodern}
\usepackage{microtype}
\usepackage{hyperref}

\author{Maxime WIRTH}
\title{Nouveaux ordres de grandeur}

\begin{document}
\maketitle

\section{Nouvelle masse}
Le diamètre du câble est de $D_{câble} = 4,950.10 mm$.\\
Pour $L = 3m$, on doit avoir 202 étages de fil et pour obtenir $3000 tr/m$, on doit avoir 15 épaisseurs de fil.\\
On peut approximer notre solénoïde par un cylindre creux de rayon intérieur $a = 1m$ et de rayon extérieur $R_{ext} = a + e$ avec $e = 15 * D_{câble} = 7.4 cm$.
On obtient un volume de $V = 1450 L$.\\
Avec $\rho = 5.7 * 10^{3} kg.m^{-3}$, on obtient finalement $M_{tot} = 8.244 t$.

\section{Les lanceurs}

\begin{tabular}{|c|c|c|}
    \hline
    Lanceur & Année de lancement ( ou dev ) & Charge utile (t) pour LEO \\
    \hline
    Long March 8 - Chine & 2020 & 7.6 \\
    \hline
    Starship - SpaceX & 2021 & 100-150 \\
    \hline
    Vulcan - USA & 2021 & 29 \\
    \hline
    Cyclone-4M - Ukraine & 2021 & 5 \\
    \hline
    Ariane 6 - UE & 2022 & 10.5 ( pour orbite de transition ) \\
    \hline
    Ariane 5 - UE & 1996 & 21 \\
    \hline
    Atlas V - USA & 2002 & 18.850 \\
    \hline
    Long March 5 - Chine & 2016 & 25 \\
    \hline
    
\end{tabular}

\end{document}