\documentclass{report}

\usepackage{global}

\graphicspath{{/home/grybouilli/TIPE/schema/}}

\lhead{Étude cinématique}

\author{Nicolas GRY}

\title{Étude du mouvement du déchet}

\begin{document}
\maketitle
\tableofcontents
\newpage
\chapter{Phase de déplacement dans le champ magnétique}
\section{Mise en situation}
\subsection{Introduction des grandeurs et notations}

On considère un solénoïde caractérisé par:
\begin{itemize}
    \item Sa longueur $L$
    \item Son rayon $a$
    \item L'intensité du courant qui le traverse $i$
    \item Son nombre de spires $n$ 
\end{itemize}
Ce solénoïde engendre un champ magnétique:
 $$\vect{B} = B_r \vect{e_r} + B_z \vect{e_z}$$.
On notera indifféremment $||\vect{B}||$ et $B$.

On considère un déchet spatiale:
\begin{itemize}
    \item modélisé par un moment magnétique solide $\vect{p}$
    \item de moment d'inertie $J_\theta$
    \item de masse m
\end{itemize} 
On notera indifféremment $||\vect{p}||$ et $p$.

On travaillera dans le repère cylindrique $(\vect{e_r},\vect{e_\theta},\vect{e_z})$ et on notera $(r,\theta,z)$ les coordonnées du déchet.

\subsection{Schématisation de la situation}

\begin{figure}[h]
    \centering
    \includegraphics[scale=0.15]{schema_etude_mouvement_1.png}
    \caption{Schéma simplifié de la situation}
\end{figure}  

On imagine que le moment magnétique $\vect{p}$ forme un angle $\alpha$ avec $\vect{e_z}$. L'objectif ici est d'étudier les variations de $\alpha$ et d'évaluer le temps que met $\vect{p}$ à s'aligner avec les lignes de champs.
\newpage
\section{Étude littérale du mouvement}
\subsection{Étude de l'alignement angulaire du déchet}

\ulbf{Système}: {débris assimilé à un moment magnétique solide $\vect{p}$ de moment d'inertie $J_\theta$}

\ulbf{Référentiel}: En lien avec le solénoïde, supposé galiléen (raisonnable au regard de la durée de l'expérience)

\ulbf{Conditions initiales}

À t = 0, on suppose que le déchet est incliné d'un angle $\alpha_0$ par rapport à l'axe de révolution du solénoïde. On suppose également la vitesse angulaire $\phder{\alpha}$ du déchet nulle à t = 0, ce qui est raisonnable puisque le déchet est dénué de tout mouvement dû au champ $\vect{B}(r,z)$ lorsqu'il n'y est pas soumis, i.e. à t = 0.

\ulbf{Bilan des forces}

\emph{NB : Il ne faut pas oublier durant cette étude que $\vect{B}$ dépend de r et de z, les variables du répère cylindrique défini dans le cadre de l'étude.}

Force magnétique :
$$\vec{F_B} = \vect{\text{grad}}(\vect{B}.\vect{p})$$

\ulbf{Théorème du moment cinétique}

Projeté selon $\vect{e_\theta}$ :
\begin{align*}
J_\theta \derived{\alpha}{t} &= \vect{\MM}(\vect{F_B}) . \vect{e_\theta} \\
&= (\vect{p} \wedge \vect{B}).\vect{e_\theta}\\
&= pB_r \cos(\alpha) - pB_z\sin(\alpha)
\end{align*}

On obtient l'équation différentielle suivante:
\begin{prettybox}[blue]
\ulbf{Équation différentielle du 2$^{nd}$ ordre non-linéaire à second membre non-constant:}
$$\derived{\alpha}{t} + \frac{pB_z}{J_\theta}\sin(\alpha) = \frac{pB_r}{J_\theta} \cos(\alpha)$$
\end{prettybox}
\subsubsection{Étude de la fin de l'alignement angulaire}
On se place en premier dans la situation où $\alpha$ est considéré petit. On a alors:
$$\derived{\alpha}{t} + \frac{pB_z}{J_\theta}\alpha = \frac{pB_r}{J_\theta} $$

On pose $$\omega_0 = \sqrt{\frac{pB_z}{J_\theta}}$$

\ulbf{Solution générale de l'équation homogène associée}

$$\alpha_h(t) = A \cos(\omega_0 t + \varphi)$$

$\, \text{ avec } A \text{ et } \varphi \text{ des constantes d'intégrations}$.

\newpage
\ulbf{Solution particulière de l'équation}

$$\alpha_p(t) = \frac{B_r}{B_z}$$


\ulbf{Solution générale de l'équation}

$$\alpha(t) = A \cos(\omega_0 t + \varphi) + \frac{B_r}{B_z}$$

\ulbf{Détermination des constantes d'intégration avec les conditions initiales}

Avec:
$$\alpha(0) = \alpha_0$$
$$\derive{\alpha}{t} (0) = 0$$

On en déduit : 
$$A = \alpha_0  - \frac{B_r}{B_z}$$
$$\varphi \equiv 0 \left[\pi\right] $$ 

Et donc finalement :

\begin{prettybox}[blue]
\ulbf{Équation de l'angle formé entre $\vect{p}$ et $\vect{e_z}$}
    $$\alpha(t) = \left(\alpha_0  - \frac{B_r(r,z)}{B_z(r,z)}\right) \cos(\omega_0 t) + \frac{B_r(r,z)}{B_z(r,z)}$$
\end{prettybox}

\subsubsection{Étude de l'alignement angulaire : Résolution numérique de l'équation différentielle}

\subsection{Étude du déplacement du moment dans le champ magnétique}

\ulbf{Système}: {débris assimilé à un moment magnétique solide $\vect{p}$}

\ulbf{Référentiel}: En lien avec le solénoïde, supposé galiléen (raisonnable au regard de la durée de l'expérience)

\ulbf{Conditions initiales}

À t = 0, on suppose que le déchet est animé d'une vitesse initiale $\vect{v_0}$.

\ulbf{Hypothèse}

On travaille d'abord sans considérer la force d'attraction gravitationnelle entre le solénoïde et le débris.

\ulbf{Bilan des forces}

\emph{NB : Il ne faut pas oublier durant cette étude que $\vect{B}$ dépend de r et de z, les variables du répère cylindrique défini dans le cadre de l'étude.}

Force magnétique d'aimantation :

\url{https://oatao.univ-toulouse.fr/17352/1/Sadowski_17352.pdf}

\end{document}