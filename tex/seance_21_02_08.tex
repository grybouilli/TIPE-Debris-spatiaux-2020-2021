\documentclass{report}

\usepackage{global}
\chead{Étude cinématique}

\author{Nicolas GRY}

\title{Étude du mouvement du déchet}

\begin{document}
\maketitle
\tableofcontents
\newpage
\chapter{Phase de déplacement dans le champ magnétique}
\section{Mise en situation}
\subsection{Introduction des grandeurs et notations}

On considère un solénoïde caractérisé par:
\begin{itemize}
    \item Sa longueur $L$
    \item Son rayon $a$
    \item L'intensité du courant qui le traverse $i$
    \item Son nombre de spires $n$ 
\end{itemize}
Ce solénoïde engendre un champ magnétique:
 $$\vect{B} = B_r \vect{e_r} + B_z \vect{e_z}$$.
On notera indifféremment $||\vect{B}||$ et $B$.

On considère un déchet spatiale, de masse $m$, que l'on modélise par un moment magnétique $\vect{p}$. 

On notera indifféremment $||\vect{p}||$ et $p$.

On travaillera dans le repère cylindrique $(\vect{e_r},\vect{e_\theta},\vect{e_z})$ et on notera $(r,\theta,z)$ les coordonnées du déchet.

\subsection{Schéma de la situation}

%à faire et à mettre dans le peudeuf

\section{Étude littérale du mouvement}
\subsection{Étude de l'alignement du déchet}

\ulbf{Système}: {débris assimilé à un moment magnétique $\vect{p}$ de masse m}

\ulbf{Référentiel}: En lien avec le solénoïde, supposé galiléen (raisonnable au regard de la durée de l'expérience)

\ulbf{Bilan des forces}

Force magnétique :
$$\vec{F_B} = \vect{\text{grad}}(\vect{B}.\vect{p})$$

\ulbf{Théorème du moment cinétique}

Projeté selon $\vect{e_\theta}$ :
\begin{align*}
m \derived{\theta}{t} &= \vect{\MM}(\vect{F_B}) . \vect{e_\theta} \\
&= (\vect{p} \wedge \vect{B}).\vect{e_\theta}\\
&= pB\sin(\theta)
\end{align*}

On obtient l'équation différentielle suivante:
$$\derived{\theta}{t} - \frac{pB}{m} \sin(\theta) = 0$$
On se place en premier dans la situation où $\theta$ est considéré petit. On a alors:
$$\derived{\theta}{t} - \frac{pB}{m} \theta = 0$$

On pose alors $$\omega_{align}^2 = - \frac{pB}{m}$$.
\end{document}