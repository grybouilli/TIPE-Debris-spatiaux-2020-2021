\documentclass[a4paper,1pt]{article}

\usepackage[french]{babel}
\usepackage[T1]{fontenc}
\usepackage[utf8]{inputenc}
\usepackage{lmodern}
\usepackage{microtype}
\usepackage{hyperref}

\title{TIPE 2020/2021 : \\ Les débris spatiaux}
\date{10 october 2020}

\begin{document}
\maketitle
\textbf{Résistance des matériaux}
-> Mécanique de milieux continus\\
Matériaux pour le filet?\\

Étudier la déchirure éventuelle du filet selon l'énergie à laquelle il serait soumis\\
$\star$ faire une étude énergétique en considérant la vitesse relative et le changement de référentiel\\
$\star$ comparer les énergies auxquelles résistent les matériaux à l'énergie trouvée avec le modèle proposé  \\

\textbf{Électro-aimants}

Checker pour faire avec un électro-aimant plutôt qu'avec un filet (car plus au programme que ce qui serait considérer avec les résistance de matériaux etc).\\

Checker le moment magnétique d'un dipôle magnétique (ici ça serait pour le débris), et les champs magnétiques induits par les solénoïdes\\
-> solution analytique pour les forces possibles mais solution uniquement numérique pour les trajectoires (faire en python une approximation des trajectoires -> éventuellement une simulation graphique)
\end{document}