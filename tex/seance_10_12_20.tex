\documentclass[a4paper,1pt]{article}

\usepackage[french]{babel}
\usepackage[T1]{fontenc}
\usepackage[utf8]{inputenc}
\usepackage{lmodern}
\usepackage{microtype}
\usepackage{hyperref}
\usepackage{amsmath}

\title{TIPE 2020/2021 : \\ Débris Spatiaux}
\date{12 octobre 2020}

\begin{document}
\maketitle

\section*{Cours sur le solénoide}
On considère un solénoide de longueur $L$ de $N$ spires. On note $n$ le nombre de spires par unité de longueur:
$$n= \frac{N}{L}$$
Pour une petite longueur d$z$, on note aussi:
$$N'=n\mathrm{d}z$$
\\
\textbf{Champ élémentaire créé sur le point M}
$$\mathrm{d} \overset{\to}{B}(M) = n\mathrm{d}z\, \frac{\mu_0 I}{2 R} \sin^3(\theta) \overset{\to}{e_z}$$

\end{document}