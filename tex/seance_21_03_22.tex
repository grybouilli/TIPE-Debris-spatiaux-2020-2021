\documentclass{article}

\usepackage{global}

\author{Maxime WIRTH}
\title{Séance 22 mars}

\begin{document}
\maketitle

\subsection{Infos}

Lancement le 22 mars d'un satellite pour tester des technologies similaires à celle pensée ici :\\
\url{https://www.ctvnews.ca/mobile/sci-tech/mission-to-clean-up-space-junk-with-magnets-set-for-launch-1.5355382}\\
Par Astroscale : \url{https://astroscale.com/}

\subsection{"Recherches"}

En changeant dans la simulation $ a : 3m \rightarrow 1m $ et $ L : 10m \rightarrow 3m $ pour avoir les ordres de grandeur des IRM, on obtient
un champ non-nul jusqu'à 2m du solénoïde : portée faible, mais faisable.

\subsection{Nico et Romain}

Calcul de la force magnétique avec le grad d'un produit scalaire, aidez-les ils ont besoin de force.

\end{document}