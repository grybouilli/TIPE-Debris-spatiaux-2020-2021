\documentclass{article}

\usepackage[french]{babel}
\usepackage[T1]{fontenc}
\usepackage{inputenc}
\usepackage{lmodern}
\usepackage{microtype}
\usepackage{hyperref}
\usepackage{amsmath}
\usepackage{graphicx}

\author{Maxime WIRTH}
\title{TIPE 2020/2021 : \\ Débris Spatiaux}
\date{12 octobre 2020}

\begin{document}
\maketitle

Explication des graphes Br0 et Bz0:

Dans notre modèle, le repère cylindrique est centré en le point situé à L/2 du solénoïde, sur l'axe de rotation.\\
De plus, on est en approximation $ r \to 0 $.

\begin{itemize}
    \item Br0:\\
        Tant qu'on est dans le solénoïde, le champ magnétique est uniforme et uniquement selon z, donc on a 0 ici.\\
        Quand z dépasse L/2, les lignes de champ "bouclent" : on a une composante selon r, qui est plus 
        grande avec r croissant, mais ça sort du modèle.
    \item Bz0:\\
        Tant qu'on est dans le solénoïde, le champ magnétique est uniforme et uniquement selon z, donc la 
        valeur est élevée.
        Quand z dépasse L/2, les lignent de champ "bouclent" : cependant, avec $r \to 0$, les LDC mettent
        un certain temps à le faire, on garde la composante.
\end{itemize}

\end{document}