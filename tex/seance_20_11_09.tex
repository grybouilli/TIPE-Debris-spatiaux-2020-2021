\documentclass[a4paper,1pt]{article}

\usepackage[french]{babel}
\usepackage[T1]{fontenc}
\usepackage[utf8]{inputenc}
\usepackage{lmodern}
\usepackage{microtype}
\usepackage{hyperref}

\title{TIPE 2020/2021 : \\ Les débris spatiaux}
\date{09 novembre 2020}

\begin{document}
 \maketitle
    Aujourd'hui:
    \begin{enumerate}
        \item Recherche des matériaux présents en des satellites
        \item Recherche sur le magnétisme
        \item Magnétons de Bohr - relation entre moment magnétique et moment cinétique
    \end{enumerate}
    ~\\

    Compte rendu avec Heyrendt le sang:

    Il existe deux (voire trois) types de matériaux magnétiques :
    \begin{enumerate}
        \item Paramagnétique (ce avec quoi on devrait finir par travailler)
        
        \emph{Matériaux qui ne possède pas d'aimantation spontanée mais qui, sous l'effet d'un champ magnétique extérieur, acquiert une aimantation orientée dans le même sens que le champ magnétique appliqué.}
        \item Diamagnétique
        
        \emph{Matériaux qui, lorsqu'il est soumis à un champ magnétique, crée une très faible aimantation opposée au champ extérieur, et génère donc un champ magnétique opposé au champ extérieur. Lorsque le champ n’est plus appliqué, l’aimantation disparaît.}
        \item (ferromagnétique) (ce avec quoi on va débuter)
        
        \emph{Matériaux qui s'aimente sous l'effet d'un champ magnétique extérieur et qui garde une partie de cette aimantation.}
    \end{enumerate}    
~\\

À faire les prochaines fois:
\begin{itemize}
    \item \textbf{faire une résolution algorithmique des équations du champ magnétique induit par un solénoide fini (cf dernier lien ajouté) en commençant avec une aimantation fixe}
    \item essayer avec une aimantation non fixe?
    \item checker la susceptibilité magnétique
    \item checker l'influence du champ gravitationnel induit par le robot (est-il négligeable?)
    \item checker l'influence des chocs sur la trajectoire du robot 
\end{itemize}

\end{document}