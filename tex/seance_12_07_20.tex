\documentclass[a4paper,1pt]{article}

\usepackage[french]{babel}
\usepackage[T1]{fontenc}
\usepackage[utf8]{inputenc}
\usepackage{lmodern}
\usepackage{microtype}
\usepackage{hyperref}
\usepackage{amsmath}
\usepackage{graphicx}

\title{TIPE 2020/2021 : \\ Débris Spatiaux}
\date{12 octobre 2020}

\begin{document}
\maketitle

Eléments orbitaux classiques:

\begin{itemize}
    \item Direction du point vernal : Droite ou vecteur passant par le centre de la Terre et qui contient le Soleil, pendant le premier jour de printemps
    \item Ligne des noeuds : Ligne qui passe par le centre de la Terre et les 2 points de l'orbite qui appartiennent au plan équitorial
    \item Noeud ascendant : Point de la ligne des noeuds par lequel l'objet traverse le plan équitorial dans la direction Sud-Nord
    \item Demi grand-axe a : la moitié de la distance périastre-apoastre
    \item Eccentricité e : Facteur qui donne l'info sur la formé générale de l'orbite (pour des objets qui orbitent autour de la terre, jamais > 1):
            \begin{itemize}
                \item e = 0 $\rightarrow$ un cercle
                \item e = 1 $\rightarrow$ une parabole
                \item e $\in ]0,1[$ $\rightarrow$ une ellipse
                \item e > 1 $\rightarrow$ une hyperbole
            \end{itemize}
    \item Inclinaison i : angle  compris entre 0 et 180°, entre le plan équitorial et la plan orbital:
            \begin{itemize}
                \item i = 90° $\rightarrow$ orbite polaire
                \item i < 90° $\rightarrow$ orbite prograde
                \item i > 90° $\rightarrow$ orbite rétrograde
            \end{itemize}
    \item Longitude du noeud ascendant $\Omega$ : Angle entre la direction du point vernal et la ligne de noeuds
    \item Argument du périgée $\omega$ : angle entre noeud ascendant et périgée de l'orbite de l'objet, dans la direction de l'orbite
    \item Anomalie vraie $\nu$ : Angle entre périgée et position de l'objet sur son orbite, dans le sens de l'orbite.
\end{itemize}

Nous avons besoin du temps de prise des données pour connaître la position d'un satellite ( l'anomalie vraie change constamment).

\end{document}