\documentclass[a4paper,1pt]{article}

\usepackage[french]{babel}
\usepackage[T1]{fontenc}
\usepackage[utf8]{inputenc}
\usepackage{lmodern}
\usepackage{microtype}
\usepackage{hyperref}

\title{TIPE 2020/2021 : \\ Les débris spatiaux}
\date{}

\begin{document}
\maketitle

\textbf{Groupe :}

\begin{itemize}
\item Romain MAURICE MP*

\item Nicolas GRY MP 

\item Maxime WIRTH MP
\end{itemize}
\vspace{1em}

\textbf{Thème :} Étude de évolution de l'impact des débris spatiaux sur la sécurité en fonction de notre gestion de ces derniers.\\


\textbf{Objectifs :}

\begin{itemize}
\item Positionner les problèmes actuels quant aux débris spatiaux ( leur nombre, l'absence de contrôle, le manque d'actions visant à les faire réduire...).

\item Montrer que ces conditions poseront dans le futur de sérieux problèmes de sécurité, tant pour les missions spatiales qu'au sol.

\item Présenter des solutions à ce problème, théoriques ou déjà tangibles, avec une ou plusieurs expérimentations techniques mettant en lumière les difficultés auxquelles il faut faire face.
\end{itemize}
\vspace{1em}

\textbf{Actions déjà accomplies :}
Recherches sur la situation des débris en orbite, sur des solutions (théoriques ou déjà faisables) pour le traitement des débris, prise de contact avec Christophe Bonnal du CNES.
\vspace{1em}

\textbf{Expérience possible:} Création d'un robot à détection de mouvement dont le but serait d'attraper un objet devant lui, l'une des solutions possibles au problème traité.


\begin{thebibliography}{11}

\bibitem[Inspiration]{Kurz1}~\\
	Kurzgesagt, \textit{End of Space - Creating a Prison for Humanity} :\\
	\url{https://www.youtube.com/watch?v=yS1ibDImAYU}\\
	
\bibitem[Informations Générales]{infos_debris}~\\
	Article d'informations générales sur les débris spatiaux, \textit{Space Debris : Facts, Removal, Research}: \\ \url{https://www.britannica.com/technology/space-debris}\\\\
    
\bibitem[Evénements]{event} ~\\

		Essais anti-satellites chinois : \url{https://en.wikipedia.org/wiki/2007_Chinese_anti-satellite_missile_test#Space_debris_tracking}\\
		
		Collision de satellites : \url{https://en.wikipedia.org/wiki/2009_satellite_collision}\\
		
		Résultat de cette collision : \url{https://ntrs.nasa.gov/archive/nasa/casi.ntrs.nasa.gov/20100002023.pdf}
		
		Quasi-rencontre de 2 stallites, janvier 2020 : \url{https://www.sciencealert.com/two-satellites-just-avoided-a-fiery-collision-how-close-did-they-come-to-disaster}\\
		
		Syndrome de Kessler : \url{https://en.wikipedia.org/wiki/Kessler_syndrome}\\


\bibitem[ClearSpace-1]{CS1}~\\
	Page du projet : \url{https://clearspace.today}\\
	
	Article de l'ESA : \url{https://www.esa.int/Safety_Security/Clean_Space/ESA_commissions_world_s_first_space_debris_removal}\\
	
\bibitem[CelesTrak]{celestrak}~\\
		Online Software de tracking des objets spatiaux : \url{https://celestrak.com/}
		\\ \url{https://celestrak.com/cesium/orbit-viz.php?tle=/pub/TLE/catalog.txt&satcat=/pub/satcat.txt&referenceFrame=1}
\bibitem[Détection]{detect}~\\
		Détection en plein jour : \url{https://www.esa.int/Safety_Security/Space_Debris/First_laser_detection_of_space_debris_in_daylight}
\\

		Michel Boër, Alain Klotz, Romain Laugier, Pascal Richard, Juan-Carlos Dolado Perez, Laurent
Lapasset, Agnès Verzeni, Sébastien Théron, David Coward, J.A. Kennewell, \textit{Tarot: A network for space surveillance and tracking operations} : \\
\url{https://conference.sdo.esoc.esa.int/proceedings/sdc7/paper/382/SDC7-paper382.pdf}\\
		
		A. Petit, E. Marchand, Keyvan Kanani. \textit{Vision-based Space Autonomous Rendezvous : A Case Study.}
IEEE/RSJ Int. Conf. on Intelligent Robots and Systems, IROS’11, 2011, San Francisco, USA, United
States. pp.619-624. ffhal-00639699f :
\url{https://hal.inria.fr/file/index/docid/639699/filename/2011_iros_petit.pdf}\\

	A. Petit, E. Marchand, K. Kanani, \textit{Tracking complex targets for space rendezvous and debris removal
applications} : \url{https://ieeexplore.ieee.org/abstract/document/6386083}\\


\bibitem[Orion]{Orion}~\\
	C.R Phipps, G.Albrecht, H.Friedman, D.Gavel, E.V.George, J.Murray, C.Ho, 
	W.Priedhorsky, M.M Michelis et J.P. Reilly, \textit{ORION: Clearing near-Earth space debris using a 20-kW, 530-nm, Earth-based, repetitively pulsed laser} :\\
	\url{https://www.cambridge.org/core/journals/laser-and-particle-beams/article/orion-clearing-nearearth-space-debris-using-a-20kw-530nm-earthbased-repetitively-pulsed-laser/9DBCF0D55220FF8073DE0FED4D339F4F}\\
	
	J.W Campbell, \textit{Project ORION: Orbital Debris Removal Using Ground-Based Sensors and Lasers} : 
	\url{https://ntrs.nasa.gov/citations/19960054373}\\
	
	Depuis l'espace : SHEN Shuangyan, JIN Xing, CHANG Hao, \textit{Cleaning space debris with a space-based laser system}\\
	\url{https://www.sciencedirect.com/science/article/pii/S1000936114001010}\\
	

\bibitem[Harpon]{harpon}~\\
		Article de l'ESA : \url{https://www.esa.int/Safety_Security/Clean_Space/Whale_of_a_target_harpooning_space_debris}\\
		
		SciNews, \textit{RemoveDEBRIS’s Harpoon captures space debris} :\\
		\url{https://www.youtube.com/watch?v=dtJ6KWPnPxo}\\
	
\bibitem[Les Tethers]{tether}~\\
		Kurzgesagt, \textit{1,000km Cable to the Stars - The Skyhook}
		\url{https://www.youtube.com/watch?v=dqwpQarrDwk&t=435s}\\

		NASA Video, \textit{Tethers Unlimited}
		\url{https://www.youtube.com/watch?v=H_bLHxqOmyE}\\
		
\bibitem[Les filets]{net}~\\
		SciNews, \textit{RemoveDEBRIS's net captures space debris} : 
		\url{https://www.youtube.com/watch?v=PIfRPTIgXuw}\\

\bibitem[HVI - HyperVelocity Impacts]{HVI}~\\
		The Royal Institution, \textit{High-Speed Collisions in Space – Experiments with a Carrot Gun} : 			\url{https://www.youtube.com/watch?v=kStpU1bU-oc}\\
		
		Image de l'ESA : \url{https://www.esa.int/ESA_Multimedia/Images/2013/04/Hypervelocity_Impact}\\
		
\bibitem[Points de Lagrange]{pl}~\\
		Yves Paumier,\textit{Les Points de Lagrange ou le démon de Kepler} : 
		\url{https://solidariteetprogres.fr/groupe-espace/points-de-lagrange/objectifs-principes-de-base/les-points-de-lagrange-ou-le-demon-de-kepler.html}\\
		
		NASA-SpacePlace, \textit{Where Do Old Satellites Go When They Die ?}
		\url{https://spaceplace.nasa.gov/spacecraft-graveyard/en/}\\
		
		Article de Luxorion : \url{http://www.astrosurf.com/luxorion/sysol-asteroides4.htm}\\

\bibitem[Illustrations/Informations complémentaires]{illu}~\\
	Site de la NASA : \url{https://www.orbitaldebris.jsc.nasa.gov/}\\
	
	ESA, \textit{ESA's Annual Space Environment Report} : \url{https://www.sdo.esoc.esa.int/environment_report/Space_Environment_Report_latest.pdf}\\
	
	NASA, Etat de l'orbite basse en 2009 : \url{https://earthobservatory.nasa.gov/images/40173/space-debris}\\
	
	K. Wormnes, R. Le Letty, L. Summerer, R. Schonenborg, O. Dubois-Matra, E. Luraschi, A. Cropp,H. Krag, and J. Delaval, \textit{ESA technologies for space debris remediation} :\\
	\url{https://www.esa.int/gsp/ACT/doc/MAD/pub/ACT-RPR-MAD-2013-04-KW-CleanSpace-ADR.pdf}\\
	
	UCS Satelitte Database : \url{https://www.ucsusa.org/resources/satellite-database#.XBEctS1oTRY}\\

\bibitem[Matériaux]{mat}
	Spacesuit : \url{https://specialtyfabricsreview.com/2018/05/01/developing-nasas-next-generation-spacesuit}\\
	
	Déformation élastique :\url{https://fr.wikipedia.org/wiki/D%C3%A9formation_%C3%A9lastique}\\

	Matériaux en général : \url{https://www.spacematdb.com}
\end{thebibliography}

\end{document}
