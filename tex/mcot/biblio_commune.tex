\documentclass{report}
\usepackage{hd_global}


\begin{document}
\section{Modélisation magnétique pour le ralliement de débris spatiaux}

La neutralisation des débris spatiaux en orbite autour de la Terre répresente un enjeu majeur pour les futures missions spatiales prévues dans les prochaines années. Des solutions se reposant sur le retrait par aimantation des débris peuvent être envisagées.

Nous proposons donc un modèle dans lequel un débris spatial pourrait être assimilé à un dipôle magnétique et pourrait être capté par un aimant modélisé par un solénoide fini. Il s'agit donc d'évaluer la puissance du champ magnétique généré par un solénoide fini et son efficacité sur un matériaux de type paramagnétique. 
\section{Bibliographie commune}

\end{document}