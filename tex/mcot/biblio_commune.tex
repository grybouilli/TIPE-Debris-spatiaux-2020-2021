\documentclass[a4paper,1pt]{article}
\usepackage{global}

\begin{document}
\section{Positionnement thématique}
\textbf{Physique ondulatoire} (Électromagnétisme), \textbf{Informatique Pratique} (Modélisation Informatique), \textbf{Physique des matériaux }(Paramagnétisme)

\section{Mots clés}
\begin{multicols}{2}
\textbf{Français}
\begin{itemize}
	\item Débris spatiaux
	\item Induction magnétique
	\item Solénoide fini
	\item Paramagnétisme
	\item Modélisation informatique
\end{itemize}
\columnbreak

\textbf{Anglais}
\begin{itemize}
	\item Space debris
	\item Magnetic induction
	\item Finite Solenoide
	\item Paramagnetism
	\item Computational modeling
\end{itemize}
\end{multicols}


\section{Modélisation magnétique pour le retrait de débris spatiaux}

La neutralisation des débris spatiaux en orbite autour de la Terre répresente un enjeu majeur pour les futures missions spatiales prévues dans les prochaines années notamment par la Nasa et SpaceX. Des solutions se reposant sur le retrait par aimantation des débris peuvent être envisagées.

Nous proposons donc un modèle dans lequel un débris spatial pourrait être assimilé à un dipôle magnétique et pourrait être capté par un aimant modélisé par un solénoide fini. Il s'agit donc d'évaluer la puissance du champ magnétique généré par un solénoide fini et son efficacité sur un matériaux de type paramagnétique. 

\section{Bibliographie commune}
Depuis la toute première mission spatiale de 1957, force est de constater qu'à chaque mise en orbite de satellites ou envoi de fusées, de multiples pièces et sections d'objets sont abandonnées pendant la lancée et se retrouvent en orbite non controlée. Plus de 500 000 débris spatiaux seraient enregistrés par la Nasa et à peu près autant seraient non détectables de par leur taille. Leurs vitesses de l'ordre de la dizaine de kilomètres par seconde en font un danger premier pour tous les objets en orbites dont nous faisons l'utilisation quotidienne comme les satellites permettant la localisation GPS, les télécommunications et les prévisions météorologiques. \cite{Kurz1}

Des accidents comme la collision enregistrée en Février 2009 entre les deux satellites Iridium 33 et Cosmos 2251 \cite{CollExemple} font le parfait exemple de l'enjeu d'une meilleure gestion des débris spatiaux. Cette collision qui a détruit un satellite parfaitement fonctionnel avec l'impact à un satellite hors service a notamment provoqué la production de près de deux milliers de nouveaux débris. 

Quelques solutions \cite{SolExemples} pour gérer les déchets spatiaux ont déjà été envisagées; d'une part, il s'agirait de les désorbiter. On identifie alors deux types de technologies: celles se basant sur une pratique d'interception de l'objet et celles reposant sur une poussée de l'objet hors de son orbite, notamment par exemple vers l'océan. 

D'autre part, il s'agirait également de pouvoir prévenir à l'avenir, une surproduction de déchet spatiaux ou du moins, de pouvoir controler cette production de rendre la gestion des débris pour facile pour les générations à venir. Dans cette perspective, la compagnie Astroscale a débuté un premier test sur un système de docking en Mars 2021 \cite{Astroscale} consistant à récupérer un satellite mis en orbite, comportant une plaque magnétique qui permettrait de l'intercepter. Cependant, cette solution ne permettrait pas de traiter les déchets déjà présents en orbite terrestre.

Ainsi, puisqu'à peu près tous les objets spatiaux en orbite sont composées d'alliages métalliques, il est possible d'imaginer un modèle reposant sur l'interception des débris spatiaux par aimantation de ces derniers. Les métaux sont en effets des matériaux paramagnétiques, auxquels on peut induire des propriétés d'aimantations sous l'effet d'un champ magnétique assez puissant \cite{MatMagnetiques}.Si à l'échelle macroscopique, la résultante des moments dipolaires élémentaires est un moment dipolaire global nul par la désorganisation des orientations des petits moments, au contact d'un champ magnétique extérieur, les moments élémentaires s'alignent avec les lignes du champ excitateur et forment un moment dipolaire macroscopique non nul. On peut alors espérer aimanter des matériaux dès lors que leurs propriétés microscopiques permet d'induire un moment dipolaire à l'échelle macroscopique.

Il s'agit donc d'évaluer l'aimantation que l'on peut induire dans un matériau qui n'en aurait à priori pas spontanément par l'étude du phénomène du paramagnétisme.

Il faut alors également imaginer un moyen de produire un champ magnétique extérieur assez puissant pour induire une telle aimantation. Un des outils les plus répandus à cet effet est le solénoide fini traversé par un courant de plus ou moins grande intensité \cite{Solenoide}. Il faut alors étudier les équations qui régissent le champ induit par un tel dispositif en prenant en compte les effets de bord habituellement négligés dans des approches simplifiées comme celle du solénoide infini. 

En modélisant numériquement ces équations du champ magnétique, il est possible d'évaluer la puissance des lignes de champs et leur portée pour diverses tailles de solénoide et ordres de grandeur de courant utilisés.
\section{Références bibliographiques}
\begin{thebibliography}{11}
\bibitem[1]{Kurz1}
	Kurzgesagt, \textit{End of Space - Creating a Prison for Humanity} :
	\url{https://www.youtube.com/watch?v=yS1ibDImAYU}

\bibitem[2]{CollExemple}
    Nasa, \textit{The Collision of Iridium 33 and Cosmos 2251: The Shape of Things to Come} : \url{https://ntrs.nasa.gov/api/citations/20100002023/downloads/20100002023.pdf}

\bibitem[3]{SolExemples}
K. Wormnes, R. Le Letty, L. Summerer, R. Schonenborg, O. Dubois-Matra, E. Luraschi, A. Cropp,H. Krag, and J. Delaval, \textit{ESA technologies for space debris remediation} :
	\url{https://www.esa.int/gsp/ACT/doc/MAD/pub/ACT-RPR-MAD-2013-04-KW-CleanSpace-ADR.pdf}\\

\bibitem[4]{Astroscale} Elsa-D, Astroscale, \url{https://astroscale.com/elsa-d/}

\bibitem[5]{MatMagnetiques}
Cours sur les matériaux magnétiques, \textit{Les Matériaux Magnétiques} : \url{https://dossier.univ-st-etienne.fr/destoucn/www/Enseignements/CMmagn%C3%A9tismeND.pdf}

\bibitem[6]{Solenoide}
Edmund E.Callaghan and Stephen H.Maslen, \textit{The Magnetic Field of a Finite Solenoide} : \url{https://ntrs.nasa.gov/api/citations/19980227402/downloads/19980227402.pdf}
\end{thebibliography}
\end{document}