\documentclass[a4paper,1pt]{article}
\usepackage{global}

\begin{document}
\section{Modélisation magnétique pour le retrait de débris spatiaux}

La neutralisation des débris spatiaux en orbite autour de la Terre répresente un enjeu majeur pour les futures missions spatiales prévues dans les prochaines années notamment par la Nasa et SpaceX. Des solutions se reposant sur le retrait par aimantation des débris peuvent être envisagées.

Nous proposons donc un modèle dans lequel un débris spatial pourrait être assimilé à un dipôle magnétique et pourrait être capté par un aimant modélisé par un solénoide fini. Il s'agit donc d'évaluer la puissance du champ magnétique généré par un solénoide fini et son efficacité sur un matériaux de type paramagnétique. 
\section{Bibliographie commune}
Depuis les toutes premières missions spatiales, force est de constater qu'à chaque mise en orbite de satellites ou envoi de fusées, de multiples pièces et sections des objets sont abandonnées pendant la lancée et se retrouvent en orbite non controlée. 
\section{Références bibliographiques}
\begin{thebibliography}{11}
\bibitem[1]{Kurz1}
	Kurzgesagt, \textit{End of Space - Creating a Prison for Humanity} :
	\url{https://www.youtube.com/watch?v=yS1ibDImAYU}
\end{thebibliography}
\end{document}