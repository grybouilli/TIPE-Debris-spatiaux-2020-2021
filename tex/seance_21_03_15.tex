\documentclass{article}

\usepackage{global}

\author{Maxime WIRTH}
\title{Séance 15 mars}

\begin{document}
\maketitle

On s'intéresse aux matériaux supraconducteurs :
Exemple : NbTi, grandement utlisé.
On recherche la masse de matériau nécassaire pour construire le solénoïde, d'un rayon de 3m et d'une longeur de 10m.

Source : \url{https://cds.cern.ch/record/2253655?ln=en}

La densité du NbTi est de $5.7 g.cm^{-3} = 5.7*10^3 kg.m^{-3}$.

Sa température critique est de $4.2K$.

Avec des filaments de $6 {\mu}m$ de diamètre, et 6400 filaments/brin, les brins font $8.3 * 10^{-4} m$, //
et  avec 36 brins/câble, on a un câble de rayon $Rc = 2.475*10^{-3} m$, donc un diamètre de $Dc = 4.950*10^{-3}$

En enroulant un câble, il nous faudrait 2020 tours pour faire 10m de hauteur.

Donc, pour obtenir les 300 000 tours sur 10m, ils nous faut 148 couches de câble, ce qui équivaut à une épaisseur de câble de
$0.7326 m$.

Finalement, avec une masse volumique de $m_v = 5.7*10^3 kg.m^{-3}$, cela nous fait une masse de $M_{tot} = 833t$.

\end{document}