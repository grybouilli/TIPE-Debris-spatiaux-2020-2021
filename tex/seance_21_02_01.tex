\documentclass{article}

\usepackage[french]{babel}
\usepackage[T1]{fontenc}
\usepackage[ utf8]{ inputenc }
\usepackage{lmodern}
\usepackage{microtype}
\usepackage{hyperref}
\usepackage{amsmath}
\usepackage{amssymb}
\usepackage{amsthm}
\usepackage{amsfonts}
\usepackage{stmaryrd}
\usepackage{graphicx}
\usepackage{ textcomp }

\newcommand{\MM}{\mathcal{M}}
\newcommand{\valmoy}[1]{\mbox{\textlangle} #1\mbox{\textrangle}}
\author{Nicolas GRY}

\title{Modélisation dipolaire d'un matériau paramagnétique}

\begin{document}
\maketitle
\section{Objectif}
Modéliser un matériau paramagnétique comme un dipole magnétique pour étudier son mouvement dans un champs magnétique.

\section{Introduction à la modélisation dipolaire}
\subsection{Momant dipolaire d'un atome}
Un électron qui gravite autour du noyau d'un atome, par son mouvement, provoque un \emph{moment magnétique orbital}. De plus, un électron comporte un \emph{moment magnétique de spin}. Le \emph{moment magnétique atomique} est donc la somme de tous les moments magnétiques orbitaux et tous les moments magnétiques de spin.

\textbf{Moment magnétique atomique}
$$\MM = \displaystyle \sum \MM_{o} + \sum \MM_{s}$$

\small 
\textbf{$\MM_{o}$}: les moments magnétiques orbitaux

\textbf{$\MM_{s}$}: les moments magnétiques de spin
\normalsize

\textbf{Remarque}

On constate donc qu'un atome dont les couches sont saturés ne peut comporter un moment magnétique, car les moments magnétiques de spin s'annulent.

\subsection{Cas du matériau paramagnétique}
Un matériau paramagnétique, hors de tout champ magnétique, n'a pas de moment magnétique. En effet, les moments magnétiques atomiques étant "désorganisés", le moment magnétique total du matériau est globalement nul.

Soumis à un champ magnétique, les moments magnétiques atomiques s'alignent selon le champs magnétique provoquant un momemnt magnétique global. On peut donc assimiler le matériau à ce moment magnétique $\MM$.

\subsection{Aimantation}
L'aimantation est définie comme étant la densité volumique de moment magnétique, c'est-à-dire: $$A = \frac{\mathrm{d}m}{\mathrm{d}V}$$
où $\mathrm{d}m$ est le moment magnétique contenu dans $\mathrm{d}V$.

Dans le cas de notre modélisation, on a simplement:
$$A = n \valmoy{m}$$ où $n$ est la densité numérique des dipôles atomiques et $\valmoy{m}$ la valeur moyenne des moments magnétiques.
\end{document}