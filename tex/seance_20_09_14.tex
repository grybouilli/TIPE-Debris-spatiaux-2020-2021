\documentclass[a4paper,1pt]{article}

\usepackage[french]{babel}
\usepackage[T1]{fontenc}
\usepackage[utf8]{inputenc}
\usepackage{lmodern}
\usepackage{microtype}
\usepackage{hyperref}

\title{TIPE 2020/2021 : \\ Les débris spatiaux}
\date{14 septembre 2020}

\begin{document}
\maketitle
\textbf{ (Idées de) Problématiques :}
\\
Comment gérer le problème que représentent les débris spatiaux?
Quel danger représentent les débris spatiaux et comment les gérer? 
\\

\textbf{Objectifs :}
\\
-Étude des populations des débris: position et nombre
\\
-Modéliser les mouvements des débris pour étudier leur impact et le potentiel de danger
\\
-Qu'est ce qui se passe si l'objet rentre dans l'atmosphère?
\\
-Étudier les différences d'impact en fonction de la forme de l'objet
\\
-Mettre en place une/des solution.s pour récupérer / détruire ces débris
\\
-Mettre en place une détection et une prédiction de la position des débris
\\
-Étudier la faisabilité des solutions proposées et leur impact à leur tour (éviter de produire plus de débris qu'il n'y en a déjà)
\end{document}