\documentclass[a4paper,1pt]{article}

\usepackage[french]{babel}
\usepackage[T1]{fontenc}
\usepackage[utf8]{inputenc}
\usepackage{lmodern}
\usepackage{microtype}
\usepackage{hyperref}
\usepackage{ansmath}

\title{TIPE 2020/2021 : \\ Débris Spatiaux}
\date{07 october 2020}

\begin{document}

\textbf{Resistance des matériaux}

$\star$ Etude énergétique rectiligne : le filet de masse m est projeté avec une accélération a vers la cible à distance d donc:\\
$ v_sortie = a \! \tau $ où $\tau$ est la durée de l'éjection (par ex avec un gaz, la durée pendant laquelle le gaz est éjecté pour propulser le filet)\\
Ainsi, en considérant v_fin constante jusqu'au contact :\\
[ Ec_r,fin = frac{1}{2}\!m\!v_fin^2 \]
L'énergie cinétique du filet selon son déplacement rectiligne vers la cible au contact\\

$\star$ Etude énergétique "interne" : le filet est initialement comprimé/"plié" dans un réservoir, et lorsqu'il est expulsé il se déploie.\\
Donc chacune des extrémités du filet tire sur le centre de ce dernier $\Rightarrow$ allongement des filins $\Rightarrow$ fragilisation à prendre en compte.\\
cf schéma "Tension-ecartement"\\

Rapide modélisation : on ne prend pas en compte la rotation de l'extrémité par rapport au centre :\\
L'éjection induit une accélération a_e à l'extrémité e de masse m_e :\\
on note $F \stackrel{def}{=} m_e \! sqrt[2]{a^2 + a_e^2}$ ; l'allongement relatif $\epsilon stackrel{def}{=} \frac{\triangle\!l}{l_0}$ où l_0 est la longueuer initiale du filin\\
et s la surface du cylindre omdélisant le filin : $s \stackrel{def}{=} 2\!\pi\!\o$ où \o est le diamètre du filin\\
Ainsi, la loi élastique donne :\\
[\epsilon = frac{F}{s\!E}\]
où E est le module de Young, qui est caractéristique du matériau utilisé.\\
On peut alors calculer cette élongation et comparer avec la résistance du matériau considéré.\\
cf schéma elongation\\

\textbf{Electro-aimants}

-> no time to do it, a faire lors de la seance du 10/12/2020

\end{document}






