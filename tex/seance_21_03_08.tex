\documentclass{article}

\usepackage{global}

\author{Maxime WIRTH}
\title{Séance 8 mars}

\begin{document}
\maketitle

On s'intéresse aux matériaux supraconducteurs :
Exemple : NbTi, grandement utlisé.
On recherche la masse de matériau nécassaire pour construire le solénoïde, d'un rayon de 3m et d'une longeur de 10m.
La densité du NbTi est de $5.7 g.cm^{-3} = 5.7*10^3 kg.m^{-3}$.
Sa température critique est de $4.2K$.


\end{document}