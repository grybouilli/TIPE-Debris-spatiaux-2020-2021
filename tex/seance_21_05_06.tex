\documentclass{article}

\usepackage{global}

\author{Nicolas GRY}

\title{Modélisation dipolaire d'un matériau paramagnétique}

\begin{document}
\maketitle
\section{Objectif}
Modéliser un matériau paramagnétique comme un dipole magnétique pour étudier son mouvement dans un champs magnétique.

\section{Introduction à la modélisation dipolaire}
\subsection{Moment dipolaire d'un atome}
Un électron qui gravite autour du noyau d'un atome, par son mouvement, provoque un \emph{moment magnétique orbital}. De plus, un électron comporte un \emph{moment magnétique de spin}. Le \emph{moment magnétique atomique} est donc la somme de tous les moments magnétiques orbitaux et tous les moments magnétiques de spin.

\textbf{Moment magnétique atomique}
\large
$$\MM = \displaystyle \sum \MM_{o} + \sum \MM_{s}$$ 
\small 

\textbf{$\MM_{o}$}: les moments magnétiques orbitaux

\textbf{$\MM_{s}$}: les moments magnétiques de spin

\normalsize

\textbf{Remarque}

On constate donc qu'un atome dont les couches sont saturés ne peut comporter un moment magnétique, car les moments magnétiques de spin s'annulent.

\subsection{Cas du matériau paramagnétique}
Un matériau paramagnétique, hors de tout champ magnétique, n'a pas de moment magnétique. En effet, les moments magnétiques atomiques étant "désorganisés", le moment magnétique total du matériau est globalement nul.

Soumis à un champ magnétique, les moments magnétiques atomiques s'alignent selon le champs magnétique provoquant un moment magnétique global. On peut donc assimiler le matériau à ce moment magnétique $\MM$.

\subsection{Aimantation}
L'aimantation est définie comme étant la densité volumique de moment magnétique, c'est-à-dire: $$M = \frac{\mathrm{d}m}{\mathrm{d}V}$$
où $\mathrm{d}m$ est le moment magnétique contenu dans $\mathrm{d}V$.

Dans le cas de notre modélisation, on a simplement:
$$M = n \valmoy{m}$$ où $n$ est la densité numérique des dipôles atomiques et $\valmoy{m}$ la valeur moyenne des moments magnétiques.

\section{Modélisation du matériau paramagnétique}
\subsection{Modèle de Langevin et modèle probabiliste (\emph{update du 10/05/21 : modèle non compatible avec les solides})}

A l’état libre, un atome est magnétique s’il est porteur de moment magnétique permanent représenté par un vecteur de module constant.
On considère un matériau paramagnétique comme un ensemble de N sites portant chacun un moment magnétique $\vec{m_i}$, tous indépendants mais de norme commune qu'on notera $||\vec{m_0} || = m_0$.

Dans le modèle de Langevin, on exprime alors d'abord le moment magnétique moyen du matériau $\valmoy{m}$ pour exprimer l'aimantation. La démonstration de Langevin de 1905 amène au résultat suivant pour l'aimantation moyenne du matériau qui sera colinéaire au champ magnétique auquel le matériau est soumis:

$$\valmoy{M_u} = Nm_0L(x)$$
où L est la fonction de Langevin telle que

$$L(x) = \coth(x) - \frac{1}{x} \mbox{ et on a posé } x = \frac{m_0 B}{k_B T}$$
\emph{N.B: L est donc une fonction de la température.}

Ce résultat est valable pour un champ magnétique qui s'exprime selon une direction $\vec{u}$. Dans notre cas, on a :

$$\vect{B} = B_r \vect{e_r} + B_z \vect{e_z} $$

donc : 
$$\vec{u} = \frac{B_r \vect{e_r} + B_z \vect{e_z}}{\sqrt{B_r^2 + B_z^2}}$$

Et on a une aimantation moyenne pour un matériau plongé dans le champ magnétique du solénoïdé fini:
\begin{prettybox}[blue]
    $$\vec{M} = Nm_0L(x) \frac{B_r(r,z) \vect{e_r} + B_z(r,z) \vect{e_z}}{\sqrt{B_r(r,z)^2 + B_z(r,z)^2}} $$
avec $$L(x) = \coth(x) - \frac{1}{x} \mbox{ et on a posé } x = \frac{m_0 \sqrt{B_r(r,z)^2 + B_z(r,z)^2}}{k_B T}$$
\end{prettybox}

\emph{Références pour les calculs : }
\begin{itemize}
    \item regarder la démo du modèle de Langevin  : \url{https://fr.wikipedia.org/wiki/Paramagn%C3%A9tisme}
    \item Infos sur la fonction de Langevin \url{https://fr.wikipedia.org/wiki/Fonction_de_Langevin}
    \item voir page 34 :  \url{https://tel.archives-ouvertes.fr/tel-01555822/document})
    \item une autre démo du modèle de Langevin : \url{http://www-chimie.u-strasbg.fr/~decomet/data/cours/magn_master_mat.pdf}
    \item calcul du moment magnétique d'un électron : \url{https://fr.wikipedia.org/wiki/Effet_Zeeman}
\end{itemize} 

\subsection{Modèle paramagnétique de Pauli pour les métaux}
Ce modèle repose sur la répartition statistique de Fermi-Dirac sur les fermions, appliquée aux électrons des atomes des métaux. 

On modélise un morceau de matériau métallique par un "gaz d'électrons". L’idée de base est considérer que les électrons sont totalement délocalisés et l’influence des noyaux et des autres électrons est moyenné de telle sorte que le potentiel vu par par un électron est constant. 

On prend pour système un parallélépipède rectangle de volume $V = l*L*h$.
L'équation de Schrödinger devient:
$$\Delta \Psi = 0$$

Pour l'étude des solides, on utilise des conditions périodiques dites de Born-von Karman, et on arrive à montrer qu'il y a discrétisation des vecteurs d'onde qui vérifient la relation:
$$E = \frac{\hbar ||\vect{k}||^2}{2m_e}$$
où $m_e$ est la masse d'un électron.

\rule{\textwidth}{0.4pt}

\begin{center}
\textbf{Définition: Énergie de Fermi}
    
    C’est l’énergie maximale atteignable par un fermion dans l’état fondamental du système de N fermions.

\emph{N.B.: Un fermion est une particule de spin demi-entier, par exemple un électron.}    
\end{center}

\rule{\textwidth}{0.4pt}

À l'énergie de Fermi $E_F$ on associe le vecteur d'onde de Fermi $\vect{k_F}$. Les niveaux d'énergie d'un atome se remplissant par ordre croissant, par définition, $\vect{k_F}$ correspond au niveau d'énergie maximal. Chaque état du système à $N_e$ électrons possède deux électrons de spins opposés et sont associés à des vecteurs d'ondes variant entre $-\vect{k_F}$ et $\vect{k_F}$. On peut donc avec les conditions de Born-von Karman montrer que :

$$\frac{N_e}{V} = \frac{k_F^3}{3\pi^2}$$

Ce qui permet d'exprimer l'énergie de Fermi pour n'importe quel gaz d'électrons:


    $$E_F(N_e) = \frac{\hbar}{2m_e} \left(\frac{3\pi^2N_e}{V}\right)^{2/3}$$
avec donc:
    \begin{itemize}
        \item $\hbar$ la constante de Planck
        \item $m_e$ la masse d'un électron
        \item $N_e$ le nombre d'électrons dans le gaz d'électron qui modélise le métal
        \item $V$ le volume de matériau considéré
    \end{itemize}
(\emph{source: }\url{https://crppwww.epfl.ch/physgen4/repository/Notes_04.05.2009.pdf})

Le paramagnétisme de Pauli donne alors le résultat suivant pour l'aimantation totale du matériau:

$$M = \frac{3\mu^2 N_e B}{2 E_F(N_e)}$$
avec:
\begin{itemize}
    \item $\mu_B$ le magnéton de Bohr
    \item $N_e$ le nombre de sites magnétiques par unité de volume
    \item $B$ la norme du champ magnétique extérieur
    \item $E_F(N_e)$ Énergie de Fermi pour un gaz de $N_e$ électrons
\end{itemize}

\subsection{Détermination de moments magnétiques pour divers métaux}
\subsubsection{Approche générale}
\url{https://www.google.com/url?sa=t&rct=j&q=&esrc=s&source=web&cd=&ved=2ahUKEwi8qsLcgsHwAhVk8-AKHWimDEsQFjAHegQICBAD&url=https%3A%2F%2Felearn.univ-oran1.dz%2Fpluginfile.php%2F55565%2Fcourse%2Foverviewfiles%2FParamagnetisme%2520de%2520Pauli.pdf%3Fforcedownload%3D1&usg=AOvVaw0VCRp_iYkn5Ij4NKFHUNlp}

Dans le doc suivant, on a un tableau avec les densité de populations d'électrons pour plusieurs éléments ce qui permet de trouver le fameux $N_e$ pour un volume défini.

(\url{https://www.google.com/url?sa=t&rct=j&q=&esrc=s&source=web&cd=&cad=rja&uact=8&ved=2ahUKEwiy1IPUosHwAhWIB2MBHUGiAtAQFjACegQIBRAD&url=http%3A%2F%2Firamis.cea.fr%2FPhocea%2Ffile.php%3Fclass%3Dcours%26file%3D%2Fcyrille.barreteau%2FPhysique_du_Solide_Barreteau_Chapitre3.pdf&usg=AOvVaw1BqPW51Q1njSGLnC4VscLL})

Maintenant, comme on est des bons gars, on va montrer quand même comment ces valeurs ont été trouvé et par chance, juste avant le tableau, y a un raisonnement rapidement détaillé lol.

D'abord, dans le modèle du gaz de fermions, notamment du gaz d'électrons, ce sont les électrons de valence qui participent à constituer le moment magnétique globale observé pour un ensemble d'atomes, les électrons de coeur étant considérés comme inertes.

On étudie la population des atomes d'une maille d'un réseau cristallin constituant le matériau considéré, on multiplie cette population par le nombre d'électrons de valence pour l'atome considéré et on obtient le nombre d'électrons actifs dans la maille. Il reste à déterminer le volume de la maille en question pour avoir la densité d'électrons pour le matériau choisi. (vive la cristallo \smiley{})
\subsubsection{Pour un matériau en aluminium}

La maille d'aluminium a une configuration cubique à face centrée de rayon atomique $r=1,43\,\AA$ donc $$4r=\sqrt{2}a \iff a = 2\sqrt{2}r \mbox{, ie :} \, a = 4,04\,\AA$$
et la maille a un volume de $v = 66,2 \,\AA^3$. L'aluminium a trois électrons de valence et la maille d'aluminium a une population de 4 donc finalement:

$$n_{Al} = 0,06 \text{ électrons} / \AA^3$$

\subsubsection{Pour un matériau en zinc}

\subsubsection{Pour un alliage Aluminium-Titane}

\subsubsection{Pour un alliage Alluminium-Cuivre}

\subsubsection{Pour un alliage Alluminium-Zinc}

\subsection{Approximations possibles selon les conditions du milieu}
Pour de hautes températures, l'énergie thermique est très supérieure à l'énergie magnétique et alors l'aimantation moyenne du matériau est très faible, de l'ordre de 0. Par contre, pour des températures proches de 0, on a l'expression donnée plus haut, $$A = n \valmoy{m}$$

\section{Problèmes possibles}
Comme le matériau à l'origine n'a pas de moment magnétique, quand on va enter en contact avec le champ du solénoïde, il va suivre les lignes de champs et se retrouver dans le solénoïde. Problème? Il risque de ressortir expulsé avec une puissance vraiment énorme.
\end{document}