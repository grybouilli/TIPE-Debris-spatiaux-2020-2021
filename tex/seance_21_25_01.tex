\documentclass{article}

\usepackage[french]{babel}
\usepackage[T1]{fontenc}
\usepackage[ utf8]{ inputenc }
\usepackage{lmodern}
\usepackage{microtype}
\usepackage{hyperref}
\usepackage{amsmath}
\usepackage{graphicx}

\author{Nicolas GRY}
\title{TIPE 2020/2021 : \\ Débris Spatiaux
\\ TP Solénoïde en tant qu'électro-aimant}

\begin{document}
\maketitle
\section{Problématiques}
\begin{itemize}
    \item Est-il réaliste d'étudier un petit solénoïde pour prévoir les effets d'un éventuel aimant réel (qu'on enverrait dans l'espace)?
    \item Quels types de matériaux sommes-nous le plus susceptibles d'atteindre?
    \item Quel ordre de grandeur pour:
    \begin{itemize}
        \item l'épaisseur du fil par rapport à l'intensité de courants (à quel moment on va faire fondre le fil)
        \item longueur et largeur du solénoïde
    \end{itemize}
\end{itemize}
\section{Objectifs}
\begin{itemize}
    \item Modéliser à petit échelle un électro-aimant
    \item Étudier le champ magnétique induit par un solénoïde et l'aimantation de différents type de matériaux
    \begin{itemize}
        \item savoir quel type de matériau on est le plus susceptible
    \end{itemize}
    \item Voir quelle configuration procure la meilleure aimantation
    \item Étudier la corrélation entre la prévision sur python et l'expérience
\end{itemize}

\section{Protocole}
\section{À faire (hors TP)}
Regarder l'expression de M l'aimantation (du déchet) en fonction de B le champ magnétique extérieur (induit par le solénoïde) $\to$ http://www.tangentex.com/ModeleIsing.htm
Problèmes
\begin{itemize}
    \item Instabilité du moment magnétique $\to$ on risque d'expulser le déchet plutôt que de le récupérer
\end{itemize}

Soluces cools
\begin{itemize}
    \item checker les supra-conducteurs
\end{itemize}
\end{document}